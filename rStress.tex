% Options for packages loaded elsewhere
% Options for packages loaded elsewhere
\PassOptionsToPackage{unicode}{hyperref}
\PassOptionsToPackage{hyphens}{url}
\PassOptionsToPackage{dvipsnames,svgnames,x11names}{xcolor}
%
\documentclass[
  12pt,
  letterpaper,
  DIV=11,
  numbers=noendperiod]{scrartcl}
\usepackage{xcolor}
\usepackage{amsmath,amssymb}
\setcounter{secnumdepth}{5}
\usepackage{iftex}
\ifPDFTeX
  \usepackage[T1]{fontenc}
  \usepackage[utf8]{inputenc}
  \usepackage{textcomp} % provide euro and other symbols
\else % if luatex or xetex
  \usepackage{unicode-math} % this also loads fontspec
  \defaultfontfeatures{Scale=MatchLowercase}
  \defaultfontfeatures[\rmfamily]{Ligatures=TeX,Scale=1}
\fi
\usepackage{lmodern}
\ifPDFTeX\else
  % xetex/luatex font selection
  \setmainfont[]{Times New Roman}
\fi
% Use upquote if available, for straight quotes in verbatim environments
\IfFileExists{upquote.sty}{\usepackage{upquote}}{}
\IfFileExists{microtype.sty}{% use microtype if available
  \usepackage[]{microtype}
  \UseMicrotypeSet[protrusion]{basicmath} % disable protrusion for tt fonts
}{}
\makeatletter
\@ifundefined{KOMAClassName}{% if non-KOMA class
  \IfFileExists{parskip.sty}{%
    \usepackage{parskip}
  }{% else
    \setlength{\parindent}{0pt}
    \setlength{\parskip}{6pt plus 2pt minus 1pt}}
}{% if KOMA class
  \KOMAoptions{parskip=half}}
\makeatother
% Make \paragraph and \subparagraph free-standing
\makeatletter
\ifx\paragraph\undefined\else
  \let\oldparagraph\paragraph
  \renewcommand{\paragraph}{
    \@ifstar
      \xxxParagraphStar
      \xxxParagraphNoStar
  }
  \newcommand{\xxxParagraphStar}[1]{\oldparagraph*{#1}\mbox{}}
  \newcommand{\xxxParagraphNoStar}[1]{\oldparagraph{#1}\mbox{}}
\fi
\ifx\subparagraph\undefined\else
  \let\oldsubparagraph\subparagraph
  \renewcommand{\subparagraph}{
    \@ifstar
      \xxxSubParagraphStar
      \xxxSubParagraphNoStar
  }
  \newcommand{\xxxSubParagraphStar}[1]{\oldsubparagraph*{#1}\mbox{}}
  \newcommand{\xxxSubParagraphNoStar}[1]{\oldsubparagraph{#1}\mbox{}}
\fi
\makeatother


\usepackage{longtable,booktabs,array}
\usepackage{calc} % for calculating minipage widths
% Correct order of tables after \paragraph or \subparagraph
\usepackage{etoolbox}
\makeatletter
\patchcmd\longtable{\par}{\if@noskipsec\mbox{}\fi\par}{}{}
\makeatother
% Allow footnotes in longtable head/foot
\IfFileExists{footnotehyper.sty}{\usepackage{footnotehyper}}{\usepackage{footnote}}
\makesavenoteenv{longtable}
\usepackage{graphicx}
\makeatletter
\newsavebox\pandoc@box
\newcommand*\pandocbounded[1]{% scales image to fit in text height/width
  \sbox\pandoc@box{#1}%
  \Gscale@div\@tempa{\textheight}{\dimexpr\ht\pandoc@box+\dp\pandoc@box\relax}%
  \Gscale@div\@tempb{\linewidth}{\wd\pandoc@box}%
  \ifdim\@tempb\p@<\@tempa\p@\let\@tempa\@tempb\fi% select the smaller of both
  \ifdim\@tempa\p@<\p@\scalebox{\@tempa}{\usebox\pandoc@box}%
  \else\usebox{\pandoc@box}%
  \fi%
}
% Set default figure placement to htbp
\def\fps@figure{htbp}
\makeatother


% definitions for citeproc citations
\NewDocumentCommand\citeproctext{}{}
\NewDocumentCommand\citeproc{mm}{%
  \begingroup\def\citeproctext{#2}\cite{#1}\endgroup}
\makeatletter
 % allow citations to break across lines
 \let\@cite@ofmt\@firstofone
 % avoid brackets around text for \cite:
 \def\@biblabel#1{}
 \def\@cite#1#2{{#1\if@tempswa , #2\fi}}
\makeatother
\newlength{\cslhangindent}
\setlength{\cslhangindent}{1.5em}
\newlength{\csllabelwidth}
\setlength{\csllabelwidth}{3em}
\newenvironment{CSLReferences}[2] % #1 hanging-indent, #2 entry-spacing
 {\begin{list}{}{%
  \setlength{\itemindent}{0pt}
  \setlength{\leftmargin}{0pt}
  \setlength{\parsep}{0pt}
  % turn on hanging indent if param 1 is 1
  \ifodd #1
   \setlength{\leftmargin}{\cslhangindent}
   \setlength{\itemindent}{-1\cslhangindent}
  \fi
  % set entry spacing
  \setlength{\itemsep}{#2\baselineskip}}}
 {\end{list}}
\usepackage{calc}
\newcommand{\CSLBlock}[1]{\hfill\break\parbox[t]{\linewidth}{\strut\ignorespaces#1\strut}}
\newcommand{\CSLLeftMargin}[1]{\parbox[t]{\csllabelwidth}{\strut#1\strut}}
\newcommand{\CSLRightInline}[1]{\parbox[t]{\linewidth - \csllabelwidth}{\strut#1\strut}}
\newcommand{\CSLIndent}[1]{\hspace{\cslhangindent}#1}



\setlength{\emergencystretch}{3em} % prevent overfull lines

\providecommand{\tightlist}{%
  \setlength{\itemsep}{0pt}\setlength{\parskip}{0pt}}



 


\usepackage{tcolorbox}
\usepackage{amssymb}
\usepackage{yfonts}
\usepackage{bm}


\newtcolorbox{greybox}{
  colback=white,
  colframe=blue,
  coltext=black,
  boxsep=5pt,
  arc=4pt}
  
\newcommand{\sectionbreak}{\clearpage}

 
\newcommand{\ds}[4]{\sum_{{#1}=1}^{#3}\sum_{{#2}=1}^{#4}}
\newcommand{\us}[3]{\mathop{\sum\sum}_{1\leq{#2}<{#1}\leq{#3}}}

\newcommand{\ol}[1]{\overline{#1}}
\newcommand{\ul}[1]{\underline{#1}}

\newcommand{\amin}[1]{\mathop{\text{argmin}}_{#1}}
\newcommand{\amax}[1]{\mathop{\text{argmax}}_{#1}}

\newcommand{\ci}{\perp\!\!\!\perp}

\newcommand{\mc}[1]{\mathcal{#1}}
\newcommand{\mb}[1]{\mathbb{#1}}
\newcommand{\mf}[1]{\mathfrak{#1}}

\newcommand{\eps}{\epsilon}
\newcommand{\lbd}{\lambda}
\newcommand{\alp}{\alpha}
\newcommand{\df}{=:}
\newcommand{\am}[1]{\mathop{\text{argmin}}_{#1}}
\newcommand{\ls}[2]{\mathop{\sum\sum}_{#1}^{#2}}
\newcommand{\ijs}{\mathop{\sum\sum}_{1\leq i<j\leq n}}
\newcommand{\jis}{\mathop{\sum\sum}_{1\leq j<i\leq n}}
\newcommand{\sij}{\sum_{i=1}^n\sum_{j=1}^n}
	
\KOMAoption{captions}{tableheading}
\makeatletter
\@ifpackageloaded{caption}{}{\usepackage{caption}}
\AtBeginDocument{%
\ifdefined\contentsname
  \renewcommand*\contentsname{Table of contents}
\else
  \newcommand\contentsname{Table of contents}
\fi
\ifdefined\listfigurename
  \renewcommand*\listfigurename{List of Figures}
\else
  \newcommand\listfigurename{List of Figures}
\fi
\ifdefined\listtablename
  \renewcommand*\listtablename{List of Tables}
\else
  \newcommand\listtablename{List of Tables}
\fi
\ifdefined\figurename
  \renewcommand*\figurename{Figure}
\else
  \newcommand\figurename{Figure}
\fi
\ifdefined\tablename
  \renewcommand*\tablename{Table}
\else
  \newcommand\tablename{Table}
\fi
}
\@ifpackageloaded{float}{}{\usepackage{float}}
\floatstyle{ruled}
\@ifundefined{c@chapter}{\newfloat{codelisting}{h}{lop}}{\newfloat{codelisting}{h}{lop}[chapter]}
\floatname{codelisting}{Listing}
\newcommand*\listoflistings{\listof{codelisting}{List of Listings}}
\makeatother
\makeatletter
\makeatother
\makeatletter
\@ifpackageloaded{caption}{}{\usepackage{caption}}
\@ifpackageloaded{subcaption}{}{\usepackage{subcaption}}
\makeatother
\usepackage{bookmark}
\IfFileExists{xurl.sty}{\usepackage{xurl}}{} % add URL line breaks if available
\urlstyle{same}
\hypersetup{
  pdftitle={Minimizing fStress and rStress by Majorizing Gauss-Newton},
  pdfauthor={Jan de Leeuw},
  colorlinks=true,
  linkcolor={blue},
  filecolor={Maroon},
  citecolor={Blue},
  urlcolor={Blue},
  pdfcreator={LaTeX via pandoc}}


\title{Minimizing fStress and rStress by Majorizing Gauss-Newton}
\author{Jan de Leeuw}
\date{January 29, 2026}
\begin{document}
\maketitle
\begin{abstract}
TBD
\end{abstract}

\renewcommand*\contentsname{Table of contents}
{
\hypersetup{linkcolor=}
\setcounter{tocdepth}{3}
\tableofcontents
}

\sectionbreak

\textbf{Note:} This is a working manuscript which will be
expanded/updated frequently. All suggestions for improvement are
welcome. All Rmd, tex, html, pdf, R, and C files are in the public
domain. Attribution will be appreciated, but is not required. The files
can be found at \url{https://github.com/deleeuw/rStress}

\sectionbreak

\section{Loss Functions}\label{loss-functions}

The Multidimensional Scaling (MDS) loss function fStress (Groenen, De
Leeuw, and Mathar (\citeproc{ref-groenen_deleeuw_mathar_C_95}{1995}), De
Leeuw (\citeproc{ref-deleeuw_E_17r}{2017})) is defined as
\begin{equation}
\sigma_f(x):=\frac12\sum_{k=1}^K w_k(f(\delta_k)-f(d_k(x)))^2,\label{eq-fdef}
\end{equation} with \(f\) increasing and differentiable in the open
interval \((0,+\infty)\). In \eqref{eq-fdef} the \(w_k\) are positive
\emph{weights}, the \(\delta_k\) are known \emph{dissimilarities}. The
vector \(x\) has the coordinates of an number of points in
\(\mathbb{R}^p\), and the \(d_k\) are Euclidean \emph{distances} between
pairs of these points. Metric least squares MDS minimizes fStress over
\(x\).

fStress was introduced and studied in Groenen, De Leeuw, and Mathar
(\citeproc{ref-groenen_deleeuw_mathar_C_95}{1995}). No explicit
algorithm to minimize it was given, but the paper has formulas for the
first and second derivatives. De Leeuw
(\citeproc{ref-deleeuw_E_17r}{2017}) uses the multivariate Faà di Bruno
formula to derive derivatives up to order four. These derivatives can be
used in general purpose minimization methods.

An important special case of fStress is rStress (also known as
powerStress), which is \begin{equation}
\sigma_r(x):=\frac12\sum_{k=1}^K w_k(\delta_k^r-d_k^r(x))^2\label{eq-rdef}
\end{equation} Special cases of rStress are Kruskal's stress (Kruskal
(\citeproc{ref-kruskal_64a}{1964a}), Kruskal
(\citeproc{ref-kruskal_64b}{1964b})) with \(r=1\), sstress by Takane,
Young, and De Leeuw (\citeproc{ref-takane_young_deleeuw_A_77}{1977})
with \(r=2\), and logarithmic stress with \(r\rightarrow 0\) by Ramsay
(\citeproc{ref-ramsay_77}{1977}). There have been various attempts to
extend the majorization (or MM) method for MDS (De Leeuw
(\citeproc{ref-deleeuw_C_77}{1977})) to rStress. References and links to
various unpublished reports are in De Leeuw
(\citeproc{ref-deleeuw_E_17r}{2017}). The recent smacofx package (Rusch
et al. (\citeproc{ref-rusch_deleeuw_mair_hornik_25}{In Press})) has R
code for the rstressMin() function that implements one of these
techniques.

Minimizing of either fStress or rStress over \(x\) is a metric MDS
problem. The \(\delta_k\), and consequently the \(f(\delta_k)\), are
\(K\) known numbers. In non-metric MDS the loss functions are minimized
over both \(x\) and \(\delta\), where \(\delta\) is constrained to be in
a set \(\Delta\subseteq\mathbb{R}^K\). For the ordinal version of
non-metric MDS, for example, we require
\(\delta_1\leq\cdots\leq\delta_K\). Since \(f\) is increasing, we can
write fStress simply as \begin{equation}
\sigma_f(x,\delta):=\frac12\sum_{k=1}^K w_k(\delta_k-f(d_k(x)))^2,\label{eq-nmfdef}
\end{equation} where \(\delta\) is no longer a vector of known
dissimilarities, but any vector monotone with the dissimilarities. These
transformed or scaled dissimilarities are often called
\emph{disparities}. Non-metric rStress is simply \eqref{eq-nmfdef} with
\(\smash{f(d_k(x))=d_k^r(x)}\).

In order to exclude the trivial solution with \(x=0\) and \(\delta=0\)
in addition we impose the normalization constraint \begin{equation}
\eta^2(\delta):=\frac12\sum_{k=1}^Kw_k\delta_k^2=1.\label{eq-normal}
\end{equation} This formulation of non-metric MDS can be generalized to
\(\delta\in\Delta\), where \(\Delta\) is a convex cone in
\(\mathbb{R}^K\). This means we require the disparities to be in the
intersection of the cone \(\Delta\) and the sphere \(\Sigma\) defined by
\eqref{eq-normal}.

\section{Alternating Least Squares}\label{alternating-least-squares}

The technique proposed in this paper to minimize non-metric fStress or
rStress is in the Alternating Least Squares (ALS) class. We start with
an initial estimate of \(x\). We then minimize \(\sigma_f\) over
\(\delta\) in \(\Delta\cap\Sigma\) for the current \(d(x)\). The
minimizing \(\delta\) is then used to minimize fStress over \(x\) for
the current disparities. These two steps are alternated until \(x\) and
\(\delta\) do not change any more. Starting in iteration \(\nu=0\) we
compute \begin{subequations}
\begin{align}
\delta^{(\nu)}&=\mathop{\text{argmin}}_{\delta\in\Delta\cap\Sigma}\sigma_f(x^{(\nu)},\delta),\label{eq-als1}\\
x^{(\nu+1)}&=\mathop{\text{argmin}}_x\sigma_f(x,\delta^{(\nu)})\label{eq-als2}.
\end{align}
\end{subequations} We then increase \(\nu\) by one and go into the next
iteration. And so on, until convergence.

\subsection{Normalized Cone
Regression}\label{normalized-cone-regression}

The step \eqref{eq-als1} is comparatively easy. We compute the least
squares projection of \(f(d(x))\) on the cone \(\Delta\) and then
normalize this projection to give it length one. If \(\Delta\) is the
cone of monotone vectors, as in ordinal MDS, the cone projection is
\emph{monotone regression}. The fact that projection on the intersection
of \(\Delta\) and \(\Sigma\) is equivalent to normalizing the projection
of \(\Delta\) is due to De Leeuw (\citeproc{ref-deleeuw_U_75a}{1975})
and more generally (and more rigorously) to Bauschke, Bui, and Wang
(\citeproc{ref-bauschke_bui_wang_18}{2018}). Since this step is the same
as in the standard MDS algorithms such as smacof (De Leeuw and Mair
(\citeproc{ref-deleeuw_mair_A_09c}{2009}), Mair, Groenen, and De Leeuw
(\citeproc{ref-mair_groenen_deleeuw_A_22}{2022})) we do not go into
details here, and just refer to the literature.

\subsection{Majorization}\label{majorization}

The second step, minimizing over \(x\) for fixed current \(\delta\), is
much more complicated. There is no analytic solution, as in the first
step, and minimization requires an iterative process of its own. Thus
the second step requires an infinite number of ``inner'' iterations. As
a compromise, we deviate from the strict ALS framework by not minimizing
fStress over \(x\) for fixed \(\delta\), but by merely taking a single
one of the ``inner'' iterations and merely decrease fStress. If this is
done judiciously we still obtain a convergent sequence of updates. This
is also what is done in other MDS algorithms such as smacof (De Leeuw
(\citeproc{ref-deleeuw_C_77}{1977})) and alscal (Takane, Young, and De
Leeuw (\citeproc{ref-takane_young_deleeuw_A_77}{1977})).

In smacof the inner iteration step is a majorization step, by now more
commonly known as an MM step. Briefly, we find a function
\(\kappa_f(x;y)\) such that

\begin{itemize}
\tightlist
\item
  \(\kappa_f(x,y)\leq\sigma_f(x)\) for all \(x\) and \(y\) in \(\Xi\),
  and
\item
  \(\kappa_f(y;y)=\sigma_f(y)\) for all \(y\) in \(\Xi\).
\end{itemize}

An inner iteration is of the form \begin{equation}
x^{(\mu+1)}=\mathop{\text{argmin}}_{x\in\Xi}\kappa(x,x^{(\mu)})\label{eq-inner}
\end{equation} Remember that this inner iterative process goes on within
step 2 of each ALS update. Now, from \eqref{eq-inner}, \begin{equation}
\sigma_f(x^{(\mu+1)})\leq\kappa_f(x^{(\mu+1)},x^{(\mu)})\leq\kappa_f(x^{(\mu)},x^{(\mu)})=
\sigma_f(x^{(\mu)}).\eqref{eq-sandwich}
\end{equation} \$\$

We can approximate \(f(d(x))\) near \(d(y)\) with \begin{equation}
f(d_k(x))\approx f(d_k(y))+\mathcal{D}f(d_k(y))(d_k(x)-d_k(y)).\label{eq-taylor}
\end{equation} Define \begin{equation}
\eta_f(x;y):=\sum_{k=1}^K w_k(\delta_k-f(d_k(y))-\mathcal{D}f(d_k(y))(d_k(x)-d_k(y)))^2.\label{eq-defeta}
\end{equation}

Note that \(\eta_f(x;x)=\sigma_f(x)\) for all \(x\) and if \(f\) is the
identity then \(\eta_f(x;y)=\sigma_f(x)\) for all \(x\) and \(y\).
Define \begin{subequations}
\begin{align}
\tilde w_k(y)&:=w_k\{\mathcal{D}f(d_k(y))\}^2,\\
\tilde\delta_k(y)&:=\frac{\delta_k-f(d_k(y))}{\mathcal{D}f(d_k(y))}+d_k(y).
\end{align}
\end{subequations} Then \[
\eta_f(x;y)=\sum_{k=1}^K\tilde w_k(y)(\tilde\delta_k(y)-d_k(x))^2
\] Now majorize. \[
\eta_f(x;y)=C+\sum_{k=1}^K\tilde w_k(y)d_k^2(x)-2\sum_{k=1}^K\tilde w_k(y)\tilde\delta_k(y)d_k(x)
\] Now if \(\delta_k(y)>0\) we use \[
d_k(x)\geq\frac{1}{d_k(y)}\text{tr}\ X'A_kY
\] and if \(\delta_k(y)<0\) we use \[
d_k(x)\leq\frac12\frac{1}{d_k(y)}\{d_k^2(y)+d_k^2(x)\}
\] Thus \[
\sum_{k=1}^K\tilde w_k(y)\tilde\delta_k(y)d_k(x)\geq\sum_{\tilde\delta_k(y)>0}w_k\frac{\delta_k(y)}{d_k(y)}x'A_ky+\frac12\sum_{\tilde\delta_k(y)<0}w_k\frac{\delta_k(y)}{d_k(y)}x'A_kx+C
\] \[
V:=\sum_{k=1}^K \tilde w_kA_k,\\
B(y):=\sum_{\tilde\delta_k(y)>0}w_k\frac{\delta_k(y)}{d_k(y)}A_k,\\
H(y):=\frac12\sum_{\tilde\delta_k(y)<0}w_k\frac{\delta_k(y)}{d_k(y)}A_k.
\] \[
\sigma_f(x)\leq C -2x'B(y)y+x'(V+H(y))x
\] \[
\sigma_f(x^+)\approx\eta_f(x^+;x)\leq\eta_f(x,x)=\sigma_f(x).
\]

\section{Gauss-Newton approximation to
rStress}\label{gauss-newton-approximation-to-rstress}

\begin{align}
\sigma_r(x)\approx&\sum_{k=1}^K w_k(\delta_k^r-d_k^r(y)-rd_k^{r-1}(y)(d_k(x)-d_k(y)))^2=\\
&\sum_{k=1}^K w_k\{rd_k^{r-1}(y)\}^2(\frac{\delta_k^r-d_k^r(y)}{rd_k^{r-1}(y)}-(d_k(x)-d_k(y)))^2
\end{align} Let \[
\tilde w_k(y):=r^2w_kd_k^{2(r-1)}(y)
\] and \[
\tilde\delta_k(y):=\frac{\delta_k^r-d_k^r(y)}{rd_k^{r-1}(y)}+d_k(y)=\frac{\delta_k^r+(r-1)d_k^r(y)}{rd_k^{r-1}(y)}
\] then \[
\sigma_r(x;y)\approx\sum_{k=1}^K\tilde w_k(y)(\tilde\delta_k(y)-d_k(x))^2
\] which can be minimized by majorization. Also if \(x=y\) then
\(\sigma_r(x;x)=\sigma_r(x)\).

\section{rStress}\label{rstress}

For \(r\geq 1\) we have \(\delta_k(y)\geq 0\).

For \(r=1\) we have \(\tilde w_k(y)=w_k\) and
\(\tilde\delta_k(y)=\delta_k\). This is regular smacof.

For \(r=2\) we have \(\tilde w_k(y)=4w_kd_k^2(y)\) and
\(\tilde\delta_k(y)=\frac{\delta_k^2+d_k^2(y)}{2d_k(y)}\).

\section{\texorpdfstring{Negative
\(\tilde\delta_k\)}{Negative \textbackslash tilde\textbackslash delta\_k}}\label{negative-tildedelta_k}

If some of the \(\tilde\delta_k\) are negative we may use the AM/GM
inequality for majorization as in Heiser
(\citeproc{ref-heiser_91}{1991}). For now the program gives an error.

\section{Nonmetric}\label{nonmetric}

In the metric case we have to decide if we want to fit \(d^r\) to
\(\delta\) or to \(\delta^r\). This can be handled in the R driver. In
the non-metric case it does not make a difference which of the two we
choose. \[
\sigma(x,\hat d)=
\]

\section*{References}\label{references}
\addcontentsline{toc}{section}{References}

\phantomsection\label{refs}
\begin{CSLReferences}{1}{0}
\bibitem[\citeproctext]{ref-bauschke_bui_wang_18}
Bauschke, Heinz H., Minh N. Bui, and Xianfu Wang. 2018. {``{Projecting
onto the Intersection of a Cone and a Sphere}.''} \emph{SIAM Journal on
Optimization} 28: 2158--88.

\bibitem[\citeproctext]{ref-deleeuw_U_75a}
De Leeuw, Jan. 1975. {``{A Normalized Cone Regression Approach to
Alternating Least Squares Algorithms}.''} Department of Data Theory
FSW/RUL.
\url{https://jansweb.netlify.app/publication/deleeuw-u-75-a/deleeuw-u-75-a.pdf}.

\bibitem[\citeproctext]{ref-deleeuw_C_77}
---------. 1977. {``Applications of Convex Analysis to Multidimensional
Scaling.''} In \emph{Recent Developments in Statistics}, edited by J. R.
Barra, F. Brodeau, G. Romier, and B. Van Cutsem, 133--45. Amsterdam, The
Netherlands: North Holland Publishing Company.

\bibitem[\citeproctext]{ref-deleeuw_E_17r}
---------. 2017. {``{Higher Partials of fStress. Who Needs Them ?}''}
2017.
\url{https://jansweb.netlify.app/publication/deleeuw-e-17-r/deleeuw-e-17-r.pdf}.

\bibitem[\citeproctext]{ref-deleeuw_mair_A_09c}
De Leeuw, Jan, and Patrick Mair. 2009. {``{Multidimensional Scaling
Using Majorization: SMACOF in R}.''} \emph{Journal of Statistical
Software} 31 (3): 1--30.
\url{https://www.jstatsoft.org/article/view/v031i03}.

\bibitem[\citeproctext]{ref-groenen_deleeuw_mathar_C_95}
Groenen, Patrick J. F., Jan De Leeuw, and Rudolf Mathar. 1995. {``{Least
Squares Multidimensional Scaling with Transformed Distances}.''} In
\emph{{From Data to Knowledge: Theoretical and Practical Aspects of
Classification, Data Analysis and Knowledge Organization}}, edited by W.
Gaul and D. Pfeifer. Berlin, Germany: Springer Verlag.
\url{https://jansweb.netlify.app/publication/groenen-deleeuw-mathar-c-95/groenen-deleeuw-mathar-c-95.pdf}.

\bibitem[\citeproctext]{ref-heiser_91}
Heiser, W. J. 1991. {``{A Generalized Majorization Method for Least
Squares Multidimensional Scaling of Pseudodistances that May Be
Negative}.''} \emph{Psychometrika} 56 (1): 7--27.

\bibitem[\citeproctext]{ref-kruskal_64a}
Kruskal, Joseph B. 1964a. {``{Multidimensional Scaling by Optimizing
Goodness of Fit to a Nonmetric Hypothesis}.''} \emph{Psychometrika} 29:
1--27.

\bibitem[\citeproctext]{ref-kruskal_64b}
---------. 1964b. {``{Nonmetric Multidimensional Scaling: a Numerical
Method}.''} \emph{Psychometrika} 29: 115--29.

\bibitem[\citeproctext]{ref-mair_groenen_deleeuw_A_22}
Mair, Patrick, Patrick J. F. Groenen, and Jan De Leeuw. 2022. {``{More
on Multidimensional Scaling in R: smacof Version 2}.''} \emph{Journal of
Statistical Software} 102 (10): 1--47.
\url{https://www.jstatsoft.org/article/view/v102i10}.

\bibitem[\citeproctext]{ref-ramsay_77}
Ramsay, James O. 1977. {``{Maximum Likelihood Estimation in
Multidimensional Scaling}.''} \emph{Psychometrika} 42: 241--66.

\bibitem[\citeproctext]{ref-rusch_deleeuw_mair_hornik_25}
Rusch, Thomas, Jan De Leeuw, Patrick Mair, and Kurt Hornik. In Press.
{``Flexible Multidimensional Scaling with the r Package Smacofx.''}
\emph{Journal of Statistical Software}, In Press.
\url{https://jansweb.netlify.app/publication/rusch-deleeuw-mair-hornik-a-25/rusch-deleeuw-mair-hornik-a-25.pdf}.

\bibitem[\citeproctext]{ref-takane_young_deleeuw_A_77}
Takane, Yoshio, Forrest W. Young, and Jan De Leeuw. 1977. {``Nonmetric
Individual Differences in Multidimensional Scaling: An Alternating Least
Squares Method with Optimal Scaling Features.''} \emph{Psychometrika}
42: 7--67.

\end{CSLReferences}




\end{document}
